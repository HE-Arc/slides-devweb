\hypertarget{applications-distribuuxe9es}{%
\section{Applications distribuées}\label{applications-distribuuxe9es}}

\begin{itemize}
\tightlist
\item
  Motivation : répartir l'exécution sur plusieurs machines

  \begin{itemize}
  \tightlist
  \item
    Principe : Les composants/services communiquent par le réseau
  \item
    Problèmes : Hétérogénéité systèmes, langages, \ldots{}
  \item
    Solution : Protocole générique, abstraction différences
  \item
    Exemples : RPC, RMI (java), CORBA, DCOM (MS)
  \end{itemize}
\item
  Utiliser les technologies du web, comme HTTP et XML :

  \begin{itemize}
  \tightlist
  \item
    indépendantes de la plateforme, éprouvées, largement utilisées
  \end{itemize}
\item
  Système distribué importance de l'architecture :

  \begin{itemize}
  \tightlist
  \item
    \href{https://en.wikipedia.org/wiki/Resource-oriented_architecture}{orientée
    ressource} : atome : ressource (donnée) : REST
  \item
    \href{https://fr.wikipedia.org/wiki/Architecture_orient\%C3\%A9e_services}{orientée
    service} : atome : service (traitement) : RPC (SOAP)
  \end{itemize}
\end{itemize}

\hypertarget{service-web}{%
\section{Service web}\label{service-web}}

\begin{itemize}
\tightlist
\item
  2 visions :

  \begin{itemize}
  \tightlist
  \item
    Utiliser les technos web pour développer des applis distribuées
  \item
    Accès pour une application aux services offerts aux humains
  \end{itemize}
\item
  Service web = webapp pour une autre application :

  \begin{itemize}
  \tightlist
  \item
    Webapps : pour humains, via un navigateur (HTTP + HTML)
  \item
    Services web : aux autres applications (HTTP + XML/JSON)
  \end{itemize}
\item
  Exemples :

  \begin{itemize}
  \tightlist
  \item
    \href{https://upload.wikimedia.org/wikipedia/commons/3/3f/Concept_WS.jpg}{Applications
    distribuées} pour l'entreprise
  \item
    \href{https://fr.wikipedia.org/wiki/Application_composite}{Mashups}
    d'applications web
    (\href{https://www.programmableweb.com/category/mashups}{exemples})
  \item
    Applications Facebook,
    \href{https://developers.google.com/apis-explorer/}{API Google}
  \item
    \href{https://ifttt.com/}{IFTTT}
  \end{itemize}
\item
  Consommer un service web ≠ Créer un service web
\end{itemize}

\hypertarget{soap}{%
\section{SOAP}\label{soap}}

\begin{itemize}
\tightlist
\item
  AVANT : Simple Object Access Protocol (obsolète)
\item
  Evolution de XML-RPC, format XML d'envoi de messages
\item
  Architecture Orientée Service (SOA)
\item
  Indépendant du langage et de la plateforme
\item
  Recommandation du w3c depuis 2003
\item
  SOAP = abus de langage, service web WS-* est plus exact
\item
  Spécifications
  \href{https://en.wikipedia.org/wiki/List_of_web_service_specifications}{WS-*}
  :

  \begin{itemize}
  \tightlist
  \item
    spécifications liées aux différents aspects des services web
  \item
    pour déployer un WS : au minimum SOAP + WSDL + UDDI
  \end{itemize}
\end{itemize}

\hypertarget{soap-1}{%
\section{SOAP}\label{soap-1}}

\begin{itemize}
\tightlist
\item
  Structure d'un message SOAP

  \begin{itemize}
  \tightlist
  \item
    Enveloppe, Entête, Corps, Erreurs
  \end{itemize}
\item
  Squelette :
\end{itemize}

\begin{english}

\begin{Shaded}
\begin{Highlighting}[]
\KeywordTok{\textless{}?xml}\NormalTok{ version="1.0"}\KeywordTok{?\textgreater{}}
\KeywordTok{\textless{}soap:Envelope}
\OtherTok{     xmlns:soap=}\StringTok{"http://www.w3.org/2001/12/soap{-}envelope"}
\OtherTok{     soap:encodingStyle=}\StringTok{"http://www.w3.org/2001/12/soap{-}encoding"}\KeywordTok{\textgreater{}}
  \KeywordTok{\textless{}soap:Header\textgreater{}}\NormalTok{  ... }\KeywordTok{\textless{}/soap:Header\textgreater{}}
  \KeywordTok{\textless{}soap:Body\textgreater{}}\NormalTok{  ... }
    \KeywordTok{\textless{}soap:Fault\textgreater{}}\NormalTok{ ... }\KeywordTok{\textless{}/soap:Fault\textgreater{}}
  \KeywordTok{\textless{}/soap:Body\textgreater{}}
\KeywordTok{\textless{}/soap:Envelope\textgreater{}}
\end{Highlighting}
\end{Shaded}

\end{english}

\hypertarget{soap-2}{%
\section{SOAP}\label{soap-2}}

\begin{itemize}
\tightlist
\item
  \href{http://www.w3schools.com/xml/xml_soap.asp}{Exemple}
  requête/réponse
\item
  \href{http://www.w3big.com/fr/soap/default.html\#gsc.tab=0}{Introduction
  à SOAP} (fr)
\item
  Créer un service web WS (SOAP) nécessite WSDL et UDDI :

  \begin{itemize}
  \tightlist
  \item
    SOAP : Echange de messages XML sur le réseau
  \item
    WSDL : Web Service Description Language
  \item
    UDDI : Universal Description, Discovery and Integration
  \end{itemize}
\item
  WSDL : Description des interfaces des web services
\item
  UDDI : Découverte et inscription aux services web

  \begin{itemize}
  \tightlist
  \item
    annuaire d'informations sur les services web
  \item
    annuaire d'interfaces de services web décrites en WSDL
  \end{itemize}
\item
  \href{http://www.w3schools.com/xml/xml_wsdl.asp}{Tutorial WSDL/UDDI
  w3schools}
\end{itemize}

\hypertarget{rest-representational-state-transfer}{%
\section{REST : REpresentational State
Transfer}\label{rest-representational-state-transfer}}

\begin{itemize}
\tightlist
\item
  Style d'architecture sur lequel a été bâti le web
\item
  Architecture Orientée Ressource (ROA)
\item
  Chapitre 5 de la
  \href{http://www.ics.uci.edu/~fielding/pubs/dissertation/top.htm}{thèse}
  de \href{https://fr.wikipedia.org/wiki/Roy_Fielding}{Roy T. Fielding}
  (\href{http://opikanoba.org/tr/fielding/rest/}{fr}), 2000
\item
  Parmi les
  \href{https://fr.wikipedia.org/wiki/Representational_state_transfer}{contraintes},
  une interface uniforme :

  \begin{itemize}
  \tightlist
  \item
    Identification des ressources (URI)
  \item
    Manipulation des ressources par des représentations
  \item
    Messages autodescriptifs
  \item
    Hypermédia comme moteur de l'état de l'application
  \end{itemize}
\item
  Ressource : information ou moyen d'accès

  \begin{itemize}
  \tightlist
  \item
    ex. : météo du jour, adresse ajout d'un article à un blog, \ldots{}
  \end{itemize}
\item
  Représentation : forme donnée à la ressource

  \begin{itemize}
  \tightlist
  \item
    ex. : page html, fichier PDF, image, flux RSS, fichier sonore,
    \ldots{}
  \end{itemize}
\end{itemize}

\hypertarget{rest}{%
\section{REST}\label{rest}}

\begin{itemize}
\tightlist
\item
  Principes

  \begin{itemize}
  \tightlist
  \item
    Identifier les ressources avec des URI (noms)
  \item
    Actions déterminées par des méthodes HTTP (verbes)

    \begin{itemize}
    \tightlist
    \item
      GET : READ (sûre)
    \item
      POST : CREATE
    \item
      PUT, PATCH : UPDATE (idempotente)
    \item
      DELETE : DELETE (idempotente)
    \end{itemize}
  \item
    Les liens hypertextes permettent de représenter le contenu :
    navigation
  \item
    Les types MIME determinent la représentation de la ressource
  \end{itemize}
\item
  Rappel

  \begin{itemize}
  \tightlist
  \item
    Sûreté : Etat de la ressource (contenu) inchangé
  \item
    Idempotence : plusieurs appels donnent le même résultat
  \end{itemize}
\end{itemize}

\hypertarget{rest-1}{%
\section{REST}\label{rest-1}}

\begin{itemize}
\tightlist
\item
  L'appel d'une ressource avec des verbes différents produira un
  résultat différent :
\end{itemize}

\begin{longtable}[]{@{}lllll@{}}
\toprule
Effet & Route & Verbe & URI (ressource) & Description\tabularnewline
\midrule
\endhead
& Index & GET & /blogs & Affiche la liste\tabularnewline
& New & GET & /blog/new & Affiche formulaire création\tabularnewline
C & Create & POST & /blogs & Création en DB, puis
redirection\tabularnewline
R & Show & GET & /blogs/42 & Affiche le blog 42\tabularnewline
& Edit & GET & /blogs/42/edit & Formulaire édition blog
42\tabularnewline
U & Update & PUT & /blogs/42 & MAJ en DB blog 42\tabularnewline
D & Destroy & DELETE & /blogs/42 & Suppression ne DB blog
42\tabularnewline
\bottomrule
\end{longtable}

\begin{itemize}
\tightlist
\item
  Laravel, Django, Rails, \ldots{} sont RESTful !
\end{itemize}

\hypertarget{niveaux-de-maturituxe9-de-richardson18}{%
\section{\texorpdfstring{Niveaux de maturité de
\href{http://martinfowler.com/articles/richardsonMaturityModel.html}{Richardson}}{Niveaux de maturité de Richardson}}\label{niveaux-de-maturituxe9-de-richardson18}}

\begin{itemize}
\tightlist
\item
  0: Plain Old Xml (POX)

  \begin{itemize}
  \tightlist
  \item
    Utilisation de HTTP pour faire du RPC
  \end{itemize}
\item
  1: Ressources

  \begin{itemize}
  \tightlist
  \item
    Ressources identifiées par URI
  \end{itemize}
\item
  2: Verbes HTTP

  \begin{itemize}
  \tightlist
  \item
    Respect des propriétés des verbes HTTP
  \end{itemize}
\item
  3: Hypertext As The Engine Of Application State (HATEOAS)

  \begin{itemize}
  \tightlist
  \item
    Les états suivants sont documentés dans la réponse
    (\textenglish{\texttt{\textless{}link\textgreater{}}})
  \end{itemize}
\end{itemize}

\hypertarget{soap-vs-rest}{%
\section{SOAP vs REST}\label{soap-vs-rest}}

\begin{itemize}
\tightlist
\item
  webservice : exposer son API en REST ou SOAP ?
\item
  SOAP (WS-*)

  \begin{itemize}
  \tightlist
  \item
    hérité du monde de l'entreprise
  \item
    plus de code pour manipuler la requête et générer la réponse
  \item
    plus flexible, extensible (namespace)
  \item
    valider requêtes depuis WDSL
  \item
    nécessité d'un framework (ex: nuSOAP en PHP)
  \end{itemize}
\item
  REST

  \begin{itemize}
  \tightlist
  \item
    hérité du web
  \item
    plus facile et rapide à utiliser
  \item
    plus lisible et plus compact
  \item
    maintenance plus facile
  \item
    meilleure tolérance aux pannes
  \end{itemize}
\end{itemize}

\hypertarget{pour-aller-plus-loin}{%
\section{Pour aller plus loin\ldots{}}\label{pour-aller-plus-loin}}

\begin{itemize}
\tightlist
\item
  Références

  \begin{itemize}
  \tightlist
  \item
    \href{https://www.w3.org/TR/soap/}{SOAP},
    \href{https://www.w3.org/2002/ws/desc/}{WSDL},
    \href{http://uddi.xml.org/}{UDDI},
    \href{http://www.ics.uci.edu/~fielding/pubs/dissertation/top.htm}{REST},
    \href{http://www.oasis-ws-i.org/}{The WSIO}
  \item
    \href{https://larlet.fr/david/biologeek/archives/20070629-architecture-orientee-ressource-pour-faire-des-services-web-restful/}{Des
    services web RESTful},
    \href{https://web.archive.org/web/20160310205502/http://home.ccil.org/~cowan/restws.pdf}{Une
    apologie de REST} (recommandés)
  \item
    \href{http://www.figer.com/Publications/SOA.htm}{REST et
    architectures orientées service},
    \href{http://fr.slideshare.net/samijaber/symposium-dng-2008-roa}{Présentation
    ROA}
  \item
    \href{http://restcookbook.com/}{The RESTful cookbook}, How important
    is
    \href{http://stackoverflow.com/questions/20335967/how-useful-important-is-rest-hateoas-maturity-level-3}{HATEOAS}
    (stack overflow)
  \end{itemize}
\item
  Exemples de services web :

  \begin{itemize}
  \tightlist
  \item
    \href{https://developers.google.com/products/}{Google},
    \href{https://developer.yahoo.com/everything.html}{Yahoo},
    \href{https://www.flickr.com/services/api/}{Flickr},
    \href{https://dev.twitter.com/overview/api}{Twitter}, \ldots{}
  \item
    \href{https://apiary.io/}{APIary} : Aide au design d'une API REST
  \item
    Tests : Postman, \href{https://hoppscotch.io/}{Hoppscotch},
    \href{https://ping-api.com/}{Ping-API},
    \href{https://testsigma.com/blog/postman-alternatives/}{autres}
  \end{itemize}
\item
  \href{http://graphql.org/}{GraphQL}

  \begin{itemize}
  \tightlist
  \item
    est destiné à devenir la prochaine évolution des apis REST utilisant
    JSON. Initié par Facebook, Github permet également d'en
    \href{https://developer.github.com/v4/}{faire usage}.
  \end{itemize}
\end{itemize}

\hypertarget{sources}{%
\section{Sources}\label{sources}}
