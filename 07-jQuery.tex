\hypertarget{jquery}{%
\section{jQuery}\label{jquery}}

\begin{itemize}
\tightlist
\item
  John Resig, 2006
\item
  Bibliothèque JS, gratuit, OS (licence MIT)
\item
  Facilite le développement JS pour les tâches fréquentes :

  \begin{itemize}
  \tightlist
  \item
    Manipulations DOM
  \item
    Manipulations CSS
  \item
    Réponse aux évenements du navigateur
  \item
    Effets visuels et animations
  \item
    Requêtes et réponses Ajax
  \end{itemize}
\item
  Abstraction implémentations différents navigateurs
\item
  Facile à apprendre
\item
  Utilisation du chaînage des méthodes et des callbacks
\end{itemize}

\hypertarget{utilisation}{%
\section{Utilisation}\label{utilisation}}

\begin{itemize}
\tightlist
\item
  Inclusion \href{https://jquery.com/download/\#other-cdns}{CDN}
\end{itemize}

\begin{english}

\begin{Shaded}
\begin{Highlighting}[]
\OperatorTok{\textless{}}\NormalTok{script src}\OperatorTok{=}\StringTok{"https://code.jquery.com/jquery{-}3.1.1.min.js"}\OperatorTok{\textgreater{}\textless{}/}\NormalTok{script}\OperatorTok{\textgreater{}}
\end{Highlighting}
\end{Shaded}

\end{english}

\begin{itemize}
\tightlist
\item
  Nos scripts
\end{itemize}

\begin{english}

\begin{Shaded}
\begin{Highlighting}[]
\OperatorTok{\textless{}}\NormalTok{script src}\OperatorTok{=}\StringTok{"application.js"}\OperatorTok{\textgreater{}\textless{}/}\NormalTok{script}\OperatorTok{\textgreater{}}
\end{Highlighting}
\end{Shaded}

\end{english}

\begin{itemize}
\tightlist
\item
  Syntaxe basique
\end{itemize}

\begin{english}

\begin{Shaded}
\begin{Highlighting}[]
\FunctionTok{$}\NormalTok{(selecteur)}\OperatorTok{.}\FunctionTok{action}\NormalTok{()}\OperatorTok{;}      \CommentTok{// $() est un raccourci pour jQuery()}
\end{Highlighting}
\end{Shaded}

\end{english}

\begin{itemize}
\tightlist
\item
  Utilisation de sélecteurs CSS, id ou classes
\end{itemize}

\begin{english}

\begin{Shaded}
\begin{Highlighting}[]
\FunctionTok{$}\NormalTok{(}\BuiltInTok{document}\NormalTok{)}\OperatorTok{;}                \CommentTok{// retourne le DOM}
\FunctionTok{$}\NormalTok{(}\StringTok{"h3"}\NormalTok{)}\OperatorTok{.}\FunctionTok{hide}\NormalTok{()}\OperatorTok{;}             \CommentTok{// cache tous les éléments h3}
\FunctionTok{$}\NormalTok{(}\StringTok{".post"}\NormalTok{)}\OperatorTok{;}                 \CommentTok{// sélectionne les éléments de classe "post"}
\KeywordTok{var}\NormalTok{ node }\OperatorTok{=} \FunctionTok{$}\NormalTok{(}\StringTok{\textquotesingle{}\textless{}p\textgreater{}New\textless{}/p\textgreater{}\textquotesingle{}}\NormalTok{)}\OperatorTok{;} \CommentTok{// un nouveau noeud}
\end{Highlighting}
\end{Shaded}

\end{english}

\begin{itemize}
\tightlist
\item
  Pour être sûr que le document est chargé :
\end{itemize}

\begin{english}

\begin{Shaded}
\begin{Highlighting}[]
\FunctionTok{$}\NormalTok{(}\BuiltInTok{document}\NormalTok{)}\OperatorTok{.}\FunctionTok{ready}\NormalTok{(}\KeywordTok{function}\NormalTok{()\{}
    \BuiltInTok{console}\OperatorTok{.}\FunctionTok{log}\NormalTok{(}\StringTok{"prêt!"}\NormalTok{)}
\NormalTok{\})}\OperatorTok{;}
\end{Highlighting}
\end{Shaded}

\end{english}

ou

\begin{english}

\begin{Shaded}
\begin{Highlighting}[]
\FunctionTok{$}\NormalTok{(}\KeywordTok{function}\NormalTok{() \{}
    \BuiltInTok{console}\OperatorTok{.}\FunctionTok{log}\NormalTok{(}\StringTok{"prêt!"}\NormalTok{)}
\NormalTok{\})}\OperatorTok{;}
\end{Highlighting}
\end{Shaded}

\end{english}

\hypertarget{suxe9lection-dans-le-dom}{%
\section{Sélection dans le DOM}\label{suxe9lection-dans-le-dom}}

\begin{itemize}
\tightlist
\item
  Sélection
\end{itemize}

\begin{english}

\begin{Shaded}
\begin{Highlighting}[]
\FunctionTok{$}\NormalTok{(}\StringTok{"h1"}\NormalTok{)}\OperatorTok{;}                        \CommentTok{// noeud élément}
\FunctionTok{$}\NormalTok{(}\StringTok{"h1"}\NormalTok{)}\OperatorTok{.}\FunctionTok{text}\NormalTok{()}\OperatorTok{;}                 \CommentTok{// noeud texte en lecture}
\end{Highlighting}
\end{Shaded}

\end{english}

\begin{itemize}
\tightlist
\item
  Modification
\end{itemize}

\begin{english}

\begin{Shaded}
\begin{Highlighting}[]
\FunctionTok{$}\NormalTok{(}\StringTok{"h1"}\NormalTok{)}\OperatorTok{.}\FunctionTok{text}\NormalTok{(}\StringTok{"Nouveau Texte"}\NormalTok{)}\OperatorTok{;} \CommentTok{// noeud texte modifié}
\end{Highlighting}
\end{Shaded}

\end{english}

\begin{itemize}
\tightlist
\item
  Tous les fils (sélecteur descendant)
\end{itemize}

\begin{english}

\begin{Shaded}
\begin{Highlighting}[]
\FunctionTok{$}\NormalTok{(}\StringTok{"\#intro li"}\NormalTok{)}\OperatorTok{;}
\end{Highlighting}
\end{Shaded}

\end{english}

\begin{itemize}
\tightlist
\item
  Que les fils directs (sélecteur d'enfants)
\end{itemize}

\begin{english}

\begin{Shaded}
\begin{Highlighting}[]
\FunctionTok{$}\NormalTok{(}\StringTok{"\#intro \textgreater{} li"}\NormalTok{)}\OperatorTok{;}
\end{Highlighting}
\end{Shaded}

\end{english}

\begin{itemize}
\tightlist
\item
  Sélecteur multiple
\end{itemize}

\begin{english}

\begin{Shaded}
\begin{Highlighting}[]
\FunctionTok{$}\NormalTok{(}\StringTok{".post, \#main "}\NormalTok{)}\OperatorTok{;}
\end{Highlighting}
\end{Shaded}

\end{english}

\begin{itemize}
\tightlist
\item
  D'autres
  \href{https://www.w3schools.com/jquery/jquery_selectors.asp}{exemples}
  de sélecteurs
\end{itemize}

\hypertarget{parcours-traversing3}{%
\section{\texorpdfstring{Parcours
(\href{https://www.w3schools.com/jquery/jquery_traversing.asp}{traversing})}{Parcours (traversing)}}\label{parcours-traversing3}}

\begin{itemize}
\tightlist
\item
  Parcours du DOM dans les trois directions :

  \begin{itemize}
  \tightlist
  \item
    Depuis le noeud courant (sélectionné)
  \item
    Haut : \textenglish{\texttt{parent(),\ parents()}}
  \item
    Bas : \textenglish{\texttt{children(),\ find()}}
  \item
    Frères : \textenglish{\texttt{sibling(),\ next(),\ prev()}}
  \end{itemize}
\item
  Filtrage

  \begin{itemize}
  \tightlist
  \item
    \textenglish{\texttt{first(),\ last(),\ eq()}}
  \item
    \textenglish{\texttt{filter(),\ not()}}
  \item
    \href{https://www.w3schools.com/jquery/jquery_ref_traversing.asp}{Référence}
  \end{itemize}
\end{itemize}

\hypertarget{modifications-de-contenu}{%
\section{Modifications de contenu}\label{modifications-de-contenu}}

\begin{itemize}
\tightlist
\item
  Accès au contenu :

  \begin{itemize}
  \tightlist
  \item
    \textenglish{\texttt{text()}} : get/set le texte entre les balises
  \item
    \textenglish{\texttt{html()}} : get/set l'élément complet (yc
    balises)
  \item
    \textenglish{\texttt{val()}} : get/set les valeurs d'un formulaire
  \item
    \textenglish{\texttt{attr()}} : set la valeur d'un attribut
  \end{itemize}
\item
  Ajout de contenu :

  \begin{itemize}
  \tightlist
  \item
    \textenglish{\texttt{append(),\ prepend()}} : au début/fin de la
    sélection (dans l'élément)
  \item
    \textenglish{\texttt{before(),\ after()}} : avant/après la sélection
  \end{itemize}
\item
  Suppression

  \begin{itemize}
  \tightlist
  \item
    \textenglish{\texttt{empty()}} : suppression des enfants
  \item
    \textenglish{\texttt{remove()}} : supression de la sélection
    (possibilité de filtrer)
  \end{itemize}
\end{itemize}

\hypertarget{accuxe8s-aux-css}{%
\section{Accès aux CSS}\label{accuxe8s-aux-css}}

\begin{itemize}
\tightlist
\item
  Accès aux classes

  \begin{itemize}
  \tightlist
  \item
    \textenglish{\texttt{addClass()}} : ajout de classe(s) à l'élément
    sélectionné
  \item
    \textenglish{\texttt{removeClass()}} : suppression de classe(s)
  \item
    \textenglish{\texttt{toggleClass()}} : suppression si présente,
    ajout sinon
  \end{itemize}
\item
  Attribut style d'un élément : \textenglish{\texttt{css()}}
\end{itemize}

\begin{english}

\begin{Shaded}
\begin{Highlighting}[]
\FunctionTok{$}\NormalTok{(}\StringTok{"p"}\NormalTok{)}\OperatorTok{.}\FunctionTok{css}\NormalTok{(}\StringTok{"background{-}color"}\NormalTok{)}\OperatorTok{;}                 \CommentTok{// get}
\FunctionTok{$}\NormalTok{(}\StringTok{"p"}\NormalTok{)}\OperatorTok{.}\FunctionTok{css}\NormalTok{(\{}\StringTok{"background{-}color"}\OperatorTok{:}\StringTok{"yellow"}\OperatorTok{,}\StringTok{"font{-}size"}\OperatorTok{:}\StringTok{"200\%"}\NormalTok{\})}\OperatorTok{;}   \CommentTok{// set}
\end{Highlighting}
\end{Shaded}

\end{english}

\hypertarget{evuxe9nements}{%
\section{Evénements}\label{evuxe9nements}}

\begin{itemize}
\tightlist
\item
  Souris

  \begin{itemize}
  \tightlist
  \item
    \textenglish{\texttt{click,\ dblclick,\ mouseenter,\ mouseleave}}
  \end{itemize}
\item
  Clavier

  \begin{itemize}
  \tightlist
  \item
    \textenglish{\texttt{keypress,\ keyup,\ keydown}}
  \end{itemize}
\item
  Formulaires

  \begin{itemize}
  \tightlist
  \item
    \textenglish{\texttt{submit,\ change,\ focus,\ blur}}
  \end{itemize}
\item
  Document

  \begin{itemize}
  \tightlist
  \item
    \textenglish{\texttt{ready,\ load,\ resize,\ scroll,\ unload}}
  \end{itemize}
\item
  Exemple
\end{itemize}

\begin{english}

\begin{Shaded}
\begin{Highlighting}[]
\FunctionTok{$}\NormalTok{(}\StringTok{"p"}\NormalTok{)}\OperatorTok{.}\FunctionTok{click}\NormalTok{(}\KeywordTok{function}\NormalTok{()\{}
  \CommentTok{// code à éxecuter ici}
\NormalTok{\})}\OperatorTok{;} 
\end{Highlighting}
\end{Shaded}

\end{english}

\hypertarget{ajax11}{%
\section{\texorpdfstring{\href{https://www.w3schools.com/jquery/jquery_ajax_load.asp}{AJAX}}{AJAX}}\label{ajax11}}

\begin{itemize}
\tightlist
\item
  \textenglish{\texttt{\$(selector).load(URL,\ data,\ callback)}}

  \begin{itemize}
  \tightlist
  \item
    \textenglish{\texttt{URL}} : Ressource ciblée par la requête
  \item
    \textenglish{\texttt{data}} : éventuel contenu
  \item
    \textenglish{\texttt{callback}} : fonction de rappel avec 3
    paramètres :

    \begin{itemize}
    \tightlist
    \item
      \textenglish{\texttt{responseTxt}}
    \item
      \textenglish{\texttt{statusTxt}}
    \item
      \textenglish{\texttt{xhr}}
    \end{itemize}
  \end{itemize}
\item
  \textenglish{\texttt{\$.get(URL,\ callback)}}
\item
  \textenglish{\texttt{\$.post(URL,\ data,\ callback)}}
\end{itemize}

\hypertarget{effets-et-animations}{%
\section{Effets et animations}\label{effets-et-animations}}

\begin{itemize}
\tightlist
\item
  \textenglish{\texttt{hide(),\ show(),\ toggle()}}
\item
  \textenglish{\texttt{fadeIn(),\ fadeOut(),\ fadeToggle()}}
\item
  \textenglish{\texttt{slideDown(),\ slideUp(),\ slideToggle()}}
\item
  \href{https://www.w3schools.com/jquery/jquery_animate.asp}{\textenglish{\texttt{animate()}}}
\end{itemize}

\hypertarget{alternatives}{%
\section{Alternatives}\label{alternatives}}

\begin{itemize}
\tightlist
\item
  \emph{jQuery aussi, ça fait vieux}, YBL 17.10.29
\item
  \href{https://gist.github.com/paulirish/12fb951a8b893a454b32}{bling.js}
\item
  API
  \href{https://www.w3schools.com/jsref/met_document_queryselectorall.asp}{queryselectorall()}
  au lieu des getElementsBy\ldots{}
\end{itemize}

\hypertarget{ruxe9fuxe9rences}{%
\section{Références}\label{ruxe9fuxe9rences}}

\begin{itemize}
\tightlist
\item
  Site officiel de \href{https://jquery.com/}{jQuery}
\item
  Tutos \href{https://www.w3schools.com/jquery/}{w3schools}
\item
  \href{https://github.com/jquery/sizzle/wiki}{SizzleJS} : uniquement
  les sélecteurs
\item
  Comparaison avec \href{http://vanilla-js.com/}{Vanilla JS}
\end{itemize}

\hypertarget{sources}{%
\section{Sources}\label{sources}}
